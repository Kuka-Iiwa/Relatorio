Em tempos de transformação digital e indústria 4.0, a robótica tem ganhado cada vez mais espaço no setor industrial. Nesse âmbito tem-se como protagonistas os robôs manipuladores que atuam sob diversas frentes nos processos de produção, tais como, por exemplo, movimentação de cargas, soldagem, pintura em \textit{spray}, esmerilhamento e afins \citep[Cap. 1]{spong08}. 

Para se realizar o controle dos robôs manipuladores deve-se obter modelos dinâmicos completos e precisos para então, com base nessas representações matemáticas, desenvolver o controle tanto do movimento livre quanto da interação com o ambiente \citep{GL17}. O primeiro passo para modelar o robô é obter a relação entre suas partes, isto é, estabelecer o quanto e como a alteração de uma variável de entrada afeta a posição e orientação do último link (\textit{end effector}). A esse processo de descrever a pose da ferramenta do robô como uma função das variáveis das juntas do mesmo, dá-se o nome de Cinemática Direta dos Robôs \citep{siciliano2010}, ou como nomeia \cite{mihelj2018}: Modelo Geométrico do Robô. 

Uma vez constatada a ampla aplicação desses robôs manipuladores, torna-se imediatamente necessário o estudo e compreensão do funcionamento dos mesmos por profissionais atuantes na área de tecnologia. Nesse contexto, foi proposto aos alunos da disciplina de Dinâmica de Robôs do curso de Engenharia Mecatrônica do CEFET-MG um estudo de caso sobre a modelagem de robôs manipuladores, aplicando a teoria de cinemática direta ao manipulador industrial KUKA LBR iiwa.

Desse modo, o presente trabalho busca descrever os procedimentos realizados para obtenção de um modelo geométrico para o robô KUKA LBR iiwa. Além dessa introdução, este texto é composto por mais quatro seções: Descrição do Robô, Modelagem do Robô pelo Método DH, Validação do Modelo, e Simulação. 

Foi criado um repositório no \texttt{GitHub} com endereço \url{https://github.com/Kuka-Iiwa/LBR-14-820} no qual contém os dados usados para a realização deste trabalho. O link do vídeo apresentando todo o desenvolvimento é \url{https://www.youtube.com/watch?v=AVyBn5LaCpY}.